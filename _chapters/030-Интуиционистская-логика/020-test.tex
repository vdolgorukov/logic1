% Autogenerated translation of 020-test.md by Texpad
% To stop this file being overwritten during the typeset process, please move or remove this header

\documentclass[12pt]{book}
\usepackage{graphicx}
\usepackage{fontspec}
\usepackage[utf8]{inputenc}
\usepackage[a4paper,left=.5in,right=.5in,top=.3in,bottom=0.3in]{geometry}
\setlength\parindent{0pt}
\setlength{\parskip}{\baselineskip}
\setmainfont{Helvetica Neue}
\usepackage{hyperref}
\pagestyle{plain}
\begin{document}

\hrule
layout: post
title: Тесты и упражнения
slug: intuitionistic2

\section*{abstract: }

\{\% include figure.html
    caption="Модель \$(M, x)\$"
    url="/assets/images/Int.png"
    class="row"
\%\}

\{\% include quiz.html 
  id="1" 
  type="multiple" 
  question="Какие из формул выполняются в \$M, x\$ на рис. выше ?" 
  options="\$p\$$\vert$\$neg p\$$\vert$\$neg neg p\$$\vert$\$p to q\$$\vert$никакие" 
  answer="3$\vert$4" 
\%\}

\{\% include figure.html
    caption="Модель \$(M, x)\$"
    url="/assets/images/Int2.png"
    class="row"
\%\}

\{\% include quiz.html 
  id="2" 
  type="multiple" 
  question="Какие из формул выполняются в \$M, x\$ на рис. выше ?" 
  options="\$p\$$\vert$\$neg p\$$\vert$\$neg neg p\$$\vert$\$q\$$\vert$\$neg q\$$\vert$\$neg neg q\$$\vert$\$p to q\$$\vert$\$q to p\$$\vert$никакие" 
  answer="9" 
\%\}

:blue\_book: \textbf{Упражнение}. Какие из указанных формул НЕ являются законами интуиционистской логики высказываний? (Постройте для таких формул контрмодели).
 1. \$neg neg p to p\$
 2. \$p to neg neg p\$
 3. \$p vee neg p\$
 4. \$neg p vee neg neg p\$
 5. \$neg (p wedge neg p)\$
 6. \$(p to q) to (neg p vee q)\$
 7. \$(neg p vee q) to (p to q)\$
 8. \$neg (p to q) to (p wedge neg q)\$
 9. \$(p wedge neg q) to neg (p to q)\$
 10. \$(p to q) to (neg q to neg p)\$
 11. \$(p to neg q) to (q to neg p)\$
 12. \$(neg p to q) to (neg q to p)\$
 13. \$(neg p to neg q) to (q to p)\$
 14. \$neg (p wedge q) to (neg p vee neg q)\$
 15. \$(neg p vee neg q) to neg (p wedge q)\$
 16. \$neg (neg p vee neg q) to (p wedge q)\$
 17. \$(p wedge q) to neg (neg p vee neg q)\$
 18. \$neg (p vee q) to (neg p wedge neg q)\$
 19. \$(neg p wedge neg q) to neg (p vee q)\$
 20. \$neg  (neg p wedge  neg q) to (p vee q)\$
 21. \$(p vee q) to neg  (neg p wedge  neg q)\$

:blue\_book: \textbf{Упражнение}. Докажите, что закон Пирса не является законом интуиционистской логики (постройте контрмодель): 
\$\$((pto q) to p) to p\$\$

:blue\_book: \textbf{Упражнение}. Найдите доказательство для слабого закона Пирса в натуральном исчислении интуиционистской логики (используя только правила для импликации): 
\$\$((((pto q) to p) to p) to q) to q\$\$

:blue\_book: \textbf{Упражнение}. Найдите результат перевода в \$S4\$ для следующих формул интуиционистской логики высказываний:
1. \$neg p\$
2. \$neg neg p\$	
3. \$p wedge neg q\$
4. \$p to q\$
5. \$p to neg neg p\$
6. \$neg neg p to p\$

\end{document}
